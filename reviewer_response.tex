\documentclass{article}
\usepackage{natbib}
\usepackage{graphicx}
\usepackage{amsmath}
\usepackage{graphicx,psfrag,epsf}
\usepackage{enumerate}
\usepackage{natbib}
\usepackage{url} % not crucial - just used below for the URL

%\pdfminorversion=4
% NOTE: To produce blinded version, replace "0" with "1" below.
\newcommand{\blind}{1}

% DON'T change margins - should be 1 inch all around.
\addtolength{\oddsidemargin}{-.5in}%
\addtolength{\evensidemargin}{-.5in}%
\addtolength{\textwidth}{1in}%
\addtolength{\textheight}{1.3in}%
\addtolength{\topmargin}{-.8in}%


\begin{document}

We would like to thank the two reviewers and the associate editor for insightful
and constructive comments on our work. Below, we describe changes to the paper
text and software based on this feedback.

\subsection{Reviewer 1}

\textit{This small sampling of a large body of work points to the need for a
  cleaner conceptualization and clarifications of the contribution of the work
  presented in this paper relative to prior efforts.}

We have have modified our literature review to begin with a broader discussion
of tree and time series visualization methods, before proceeding to those
methods used more directly in our work. Further, we have added a bullet list
describing more concretely the contributions of this paper within the broader
body of knowledge.

\textit{My main concern regarding the presentation of the work lies in the third
  and fourth views - why are they presented as distinct? It would seem to me
  that all of the described functionality could be configured in that one view.
  In fact, I could imagine including the DOi in the same view as well - why are
  multiple views needed?}

The idea of combining these displays is interesting, and we have attempted to
implement a version which would highlight the series / nodes that lay in both
(1) the union of all treeboxes and (2) the intersection of all timeboxes.
However, we found this idea difficult to implement, for technical rather than
conceptual reasons -- when attempting to have both brush interactions for
querying and mouseover interactions for providing detail, the events conflict.
Indeed, we implemented the first and third recommendations described in
\href{https://github.com/d3/d3/issues/1604}{this d3 issue}, but in both cases we
found it necessary to introduce a lag between mousedown and the start of
brushing in order for the event to properly register. This lag created a drag on
responsiveness, and the value gained by unifying the displays did not seem worth
the cost of less fluid interactivity.

\textit{Some detail about the scale of the tool would be useful. How many nodes
  can be included in the tree? How many time series, of what length? Rough
  estimates would help readers assess utility.}

We have added a description in Section ? describing the scales of different
inputs before the interface begins to lag.

\subsection{Reviewer 2}

\textit{In the case studies, it would be interesting to see how the highlighted
  nodes change when you shrink or pan the selecting area in the time series
  plot.}

This is visible online in our video demo. It is also possible to see this effect
directly by interacting with precompiled versions in our compiled rmarkdown
vignettes. We have further added a short supplementary note providing static
figures describing this effect.

\textit{Is it possible to combine the DOI sankey idea with the timebox trees, so
  that the time series can be conditioned on the categorical variable?}

It would be possible to colo series according to different properties, or
alternatively faceting different series into separate panels. We have these
extensions in mind for a future version of the package, though these are not
possible in the package as it currently stands.

\textit{As the authors mentioned in the conclusion part, faceting is an
  important feature to be added in the interactive time series plot. Also,
  zooming in $y$-scale is needed when the lines are very intensive, like the
  bottom of Figure 1.}

As mentioned above, faceting is a natural next step, but we have not implemented
it yet. On zooming however, it is already possible to zoom the $y$-axis, using
the pan-zoom widget.

\textit{In the case studies, when there are two colors in the time series plot
  (Figures 4, 9, 13, 16), the brown color is almost hidden by the blue. It would
  be nicer to have the brown highlighted on the top, or add the transparency to
  the blue.}

It is already possible to customize the transparency of the blue selected lines
by using the \texttt{ts\$opacity\_selected} argument in the \texttt{style\_opts}
argument to the R plotting functions. However, we realize this is not the ideal
approach to displaying the searched series, and since searching is an important
function, we have revised our implementation and the relevant figures so that
the searched series are always displayed on top.

\textit{In the clustering example, it doesn't make sense to show the internal
  nodes. It is better to display only the leaves in the time series plot.}

We actually think that the nodes are useful to plot, because they can be
interpreted as centroids of the associated subtree, which are a natural object
of study.

\textit{Page 7, Line 46, ``node sizes reflecting the time series value at that
  node''. Does that mean the node size will change when the time changes?}

Sorry, what we actually meant is that the node size reflects the
\textit{average} value across all times. This has been clarified in the text.

\textit{Page 11, lines 28-40. Should have referred to Figure 5.}

We have added a reference to Figure 5 at the point that we start describing the
series for Verrucomicrobiae.

\textit{Page 13, Line 42. ``The red edges are associated with preterm births''.
  Actually red represents term in Figure 8. Hence lines 45-47 are not correct.}

We have fixed this in the text to refer to the green and yellow edges.

\textit{Page 15, Line 46. How did you get the numbers of the power, 5.6 and 6.7?
  In Figure 13 the $y$-axis goes from around 11 to 15, and I can't find any
  number close to 6.}

We had accidentally changed the base of the log between generating this figure
and our text. The text and figure now both use base 10 logs.

\textit{Some of the analysis from the figures are not grounded. For example,
  page 14 lines 45-47 can not be concluded from Figure 9 easily. The analysis on
  page 15 lines 37-39 are not directly from the plot, because Figures 11 and 12
  themselves do not tell the county names.}

We realize this is a difficulty of explaining inferences gathered from
interactive visualizations through a sequence of static figures. In the text, we
have now noted some of the interactive gestures the user would have to make to
completely ground these analysis. For this issue of reproducibility, we have
also made compiled interactive versions of these examples avaialble so that
users can confirm the analysis for themselves, without having to install
treelapse or rerun examples.

\subsection{Associate Editor Review}

\end{document}
