\documentclass{article}
\usepackage{natbib}
\usepackage{graphicx}
\usepackage{amsmath}
\usepackage{graphicx,psfrag,epsf}
\usepackage{enumerate}
\usepackage{natbib}
\usepackage{url} % not crucial - just used below for the URL

%\pdfminorversion=4
% NOTE: To produce blinded version, replace "0" with "1" below.
\newcommand{\blind}{1}

% DON'T change margins - should be 1 inch all around.
\addtolength{\oddsidemargin}{-.5in}%
\addtolength{\evensidemargin}{-.5in}%
\addtolength{\textwidth}{1in}%
\addtolength{\textheight}{1.3in}%
\addtolength{\topmargin}{-.8in}%


\begin{document}

We would like to thank the two reviewers and the associate editor for insightful
and constructive comments on our work. Below, we describe changes to the paper
text and software based on this feedback.

\subsection{Reviewer 1}

\textit{This small sampling of a large body of work points to the need for a
  cleaner conceptualization and clarifications of the contribution of the work
  presented in this paper relative to prior efforts.}

We have have modified our literature review to begin with a broader discussion
of tree and time series visualization methods, before proceeding to those
methods used more directly in our work. Further, we have added a bullet list
describing more concretely the contributions of this paper within the broader
body of knowledge.

\textit{My main concern regarding the presentation of the work lies in the third
  and fourth views - why are they presented as distinct? It would seem to me
  that all of the described functionality could be configured in that one view.
  In fact, I could imagine including the DOi in the same view as well - why are
  multiple views needed?}

Yes, it would be possible to develop a view that highlights series based on the
intersection of (1) the union of brushes over the trees and (2) the intersection
of brushes over the time series. We imagined that it would be easier for users
to interpret the display by explicitly modularizing the separate visual tasks.
(todo: we should try to experiment with this idea, to see if it is feasible)

\textit{Some detail about the scale of the tool would be useful. How many nodes
  can be included in the tree? How many time series, of what length? Rough
  estimates would help readers assess utility.}

We have added a description in Section ? describing the scales of different
inputs before the interface begins to lag.

\subsection{Reviewer 2}

\end{document}
