%\bibliographystyle{natbib}

\def\spacingset#1{\renewcommand{\baselinestretch}%
{#1}\small\normalsize} \spacingset{1}


%%%%%%%%%%%%%%%%%%%%%%%%%%%%%%%%%%%%%%%%%%%%%%%%%%%%%%%%%%%%%%%%%%%%%%%%%%%%%%

\if0\blind
{
  \title{\bf Interactive Visualization of Hierarchically Structured Data}
  \author{Kris Sankaran\thanks{
      This research was supported in part by an NIH training grant (5T32GM096982-03).}\hspace{.2cm}\\
    Department of Statistics, Stanford University\\
    and \\
    Susan Holmes \\
    Department of Statistics, Stanford University}
  \maketitle
} \fi

\if1\blind
{
  \bigskip
  \bigskip
  \bigskip
  \begin{center}
    {\LARGE\bf Title}
\end{center}
  \medskip
} \fi

\bigskip
\begin{abstract}
We introduce methods for visualization of data structured along trees,
especially hierarchically structured collections of time series.  To
this end, we identify questions that often emerge when working with
hierarchical data and provide an R package to simplify their
investigation. Our key contribution is the adaptation of the
visualization principles of focus-plus-context and linking to the
study of tree-structured data.

Our motivating application is to the analysis of microbial time
series, where an evolutionary tree relating microbes is available a
priori. However, we have identified common problem types where, if a
tree is not directly available, it can be constructed from data and
then studied using our techniques. We perform detailed case studies to
describe the alternative use cases, interpretations, and utility of
the proposed visualization methods.

\end{abstract}

\noindent%
{\it Keywords:}  focus-plus-context, linking, time-series, tree-structured, R, D3
\vfill

\newpage
\spacingset{1.45} % DON'T change the spacing!
