\documentclass{article}
\usepackage{natbib}
\usepackage{setspace}
\usepackage{graphicx}
\usepackage{hyperref}
\usepackage{color}
\linespread{1.25}

\title{Response to Reviews: JCGS-17-033.R2}
\begin{document}
\maketitle

We would like to thank the reviewers for their encouragement and comments
leading to valuable improvements of our software and paper. The original
reviewer comments are printed in blue, and our responses are in black.

\subsection{Reviewer 1}
\label{subsec:reviewer_1}

\color{blue}
Comments to the Author

Thanks to the authors for having addressed concerns from the previous review.
The revisions made in response to these comments have, in my opinion,
strengthened the paper.

I als want to commend the details presented in the video and on the GitHub
pages. Very nicely done.

I have a few minor suggestions for a few additional revisions that might help
the paper.

1. Regarding the third and fourth view, your response to earlier comments
suggests that implementation details interfered with attempts to combine these
views. Discussing these concerns in the paper (or on the documentation website)
might help others either avoid such difficulties or identify appropriate
work-arounds. Please consider adding relevant text to the paper.

\color{black}

\color{blue}
2. On page 7, the description of the tree reads: "Every edge in the tree is
split into two colors1, with relative widths of the different colors reflecting
differences in sizes for the two groups. The overall width of each edge
represents the sum of values across all groups." I was a bit confused by these
sentences, as they seem redundant. Doesn't the second sentence follow directly
from the first? If not, some additional clarification is needed.

\color{black}
It's true, those sentences are redundant. We realize now that this can be
confusing, so we have removed the second sentence.

\color{blue}
3. Regarding the tree ordering, I found myself wondering about the reordering of
the trees based on the size of the subtree. As this seems to be to be a tradeoff
between information clarity and consistency, I can imagine some cases where this
behavior might be useful, and others where it might be distracting. Have you
considered the possibility of animating the transition, so was to help users
avoid disorientation? Alternatively, a toggle to enable/disable this behavior
might help.

\color{black}

\subsection{Reviewer 2}
\label{subsec:reviewer_2}

\color{blue}
Comments to the Author

The authors have made substantial changes to the manuscript according to the
comments from reviewers and editors.

However, I still have one concern about the display of hierarchical clustering.
The authors did answer the question about this example from the second reviewer:
"...the (internal) nodes are useful to display, because they can be interpreted
as centroids of the associated subtree..." I agree that in other examples and
some scenarios where the internal nodes serve as meaningful group dividers, like
the California home prices data, they are useful to display. But hierarchical
clustering is a different story. All the internal nodes are binary, and the
centroids of subtrees are not of interest especially for those lower-level
subtrees. Too many time series lines for the centroids may mislead or increase
the difficulty of user's understanding about the original data. For the Global
Patterns data, Figure 19 (left) highlights 11 nodes of which only 6 are the
leaves, i.e., the original samples. So there are almost half of the time series
lines are additional. Things will get worse when the subtree is unbalanced,
because the centroids will emphasize the larger subgroups and weaken the smaller
subgroups by drawing more time series lines for the larger subgroups.

To solve this problem, it would be good to provide the option to highlight
different levels of nodes. For example, if the user selects the lowest level,
only the original data points will appear.

\color{black}
We appreciate the reviewers elaboration of the problems that can arise when
plotting series associated with internal nodes, and have taken the recommended
steps to resolve the issue. In particular, for both timebox trees and treeboxes,
we have added functionality to either (1) only display time series associated
with leaves (original data samples) or (2) only display time series within a
certain window of depths on the tree.

We have updated Figure 19 (the hierarchical clustering example), so that only
series associated with leaves are shown. The vignette for this example has also
been updated, so that users can see how to access this functionality.

\end{document}
